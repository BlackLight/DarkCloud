\chapter{Introduzione}
A livello globale si pu� osservare una crescita esponenziale in termini di numero e gravit� di attacchi informatici. Il problema per� non tocca solo le grandi aziende, questo era vero una volta, perch� solo le aziende di una certa dimensione avevano server interni e un sistema informatizzato raggiungibile dell'esterno. Oggi non esiste azienda che non abbia un calcolatore collegato ad internet. Nella stragrande maggioranza dei casi si hanno uffici con reti di computer, sedi dislocate per il mondo collegate tra di loro tramite lan, server interni per rendere disponibile materiale a consulenti e molto altro ancora. Si compie un grave errore pensando che solo le aziende che trattano materiale sensibile siano in pericolo. Per esempio aziende concorrenti potrebbero voler rubare tecnologie e brevetti aziendali, consultare liste clienti e fornitori, male intenzionati potrebbero infettare le macchine per usarle come zombie nelle proprie botnet.

Oggi pi� che mai bisogna rendere consapevoli le aziende della situazione nella quale il mondo dell'informatica si sta evolvendo. 

La sicurezza delle organizzazioni pi� importanti risiede nel livello di protezione delle proprie informazioni.

Ci sono molti metodi per proteggere documenti informatici. Alcuni prevedono che il dato sia
protetto fisicamente, altri che sia protetto da password o ancora sia raggiungibile dalla rete
ma dietro solidi firewall.

Nella societ� attuale per� non ci si pu� permettere di avere importanti moli di dati
che non siano fruibili da persone autorizzate sparse per tutto il globo.
Quindi la disponibilit� in rete � un requisito fondamentale, che espone per� i dati a gravi rischi.
Riporre le proprie speranze in soluzioni monolitiche come pesanti crittografie o 
enti che raccolgono molti dati si � rivelato nel tempo una soluzione non vincente.

Dovendo destreggiarsi nella rete � conveniente sfruttare quelli che sono i suoi punti forti, 
come ad esempio l'enorme quantit� di nodi che la compongono e la semplicit� di scambio dei dati.

La condizione necessaria che un software di condivisione deve soddisfare � proteggere il file da utenti male intenzionati che potrebbero volersene appropriare. 

La soluzione a questo problema � garantita da tre fattori, frammentazione del file, darknet e cloud computing. 

Questi tre elementi non sono abitualmente usati insieme. 
Il cloud computing sta vedendo un forte momento di crescita e le sue implementazioni stanno crescendo molto. Per� molte aziende che offrono servizi online usano i dati degli utenti per ricerche di mercato e altri motivi che non tutelano la privacy. Per tale motivo non � consigliabile usare tali servizi per dati confidenziali.
Esistono strumenti di anonimato e reti di condivisione anonima che proteggono gli utilizzatori come le darknet ma sono spesso poco pratici a causa di tempi di condivisione troppo lunghi che rendono impraticabile scambi di file di grosse dimensioni o scambi frequenti con altri utenti. 
L'utilizzo della frammentazione dei file oggi � molto diffusa per aumentare le velocit� di condivisione. Questo strumento pu� anche essere utilizzato per aumentare la sicurezza. Unire questi elementi in un unico software permette invece di avere uno strumento innovativo ed estremamente sicuro. 

\section{Obiettivi del software}
L'obbiettivo primario del software � la sicurezza dei dati che l'utente condivide con gli altri utilizzatori della sua rete. Vi sono anche aspetti che si rendono necessari legati all'architettura e al funzionamento del software cio� un codice con forte modularit� e scalabilit�.

Darkcloud � un software di condivisione file, cio� un programma che installato su almeno due computer e almeno un server permette ad ogni utente di salvare file sul server e condividere il file con l'altro utente. 
Il dato viene frammentato e salvato in vari nodi server sparsi sul web. Si ricorre ad un tipo di rete molto utilizzato nel'underground digitale, che viene definita {\itshape darknet}. In questo tipo di rete solo chi ne fa parte sa da chi � composta la rete. 

Questo documento si pone l'obbiettivo di guidare il lettore in tutto il ciclo di creazione del programma Darkcloud, dallo studio fino alla fase di test.

Nel capitolo due viene analizzato lo stato in cui si trova al momento la ricerca nel campo delle reti di condivisione, inoltre viene fatta un analisi dei protocolli pi� utilizzati sottolineandone pregi e difetti. 

Nel capitolo tre vengono motivate le scelte fatte in sede di progetto, si illustrano la struttura e il funzionamento del software. Inoltre si descrive la topologia della rete che sta dietro Darkcloud. 

Il capitolo quattro contiene la descrizione di basso livello del software. Nei primi paragrafi si parla degli strumenti utilizzati come la crittografia, i database e le tecniche di connessione. Si procede all'esposizione dettagliata del funzionamento dei comandi e delle meccaniche che stanno dietro il software.

Il capitolo cinque tratta i test eseguiti al termine della fase di programmazione, comprende la descrizione degli ambienti usati per il testing, i risultati sperimentali e un analisi della scalabilit� del software. 

Il capitolo sei include considerazioni su tutto il progetto chiarendo i risultati ottenuti e la posizione in cui questo software si pone nella ricerca moderna. Include inoltre alcune prospettive per futuri sviluppi di ricerca. 