\chapter{Introduzione}
Un ladro scassina una serratura, entra in una stanza buia piena di schedari.
Inizia a frugare negli indici e trova la scheda che gli interessa.
Prende la scheda ed esce di fretta, una guardia preposta al controllo del ufficio lo insegue, ma 
il ladro pi� veloce lo semina e scappa.

Nel corso degli anni i dati sono stati sempre pi� centralizzati e il rischio di fughe di dati � 
aumentato al punto che si � reso necessario prendere
provvedimenti per la loro sicurezza.

La tecnologia ha fornito strumenti sempre pi� efficaci per proteggere i propri dati, sistemi di 
sorveglianza, sensori vari, tutto atto alla sicurezza fisica .
Poi sono nati i primi calcolatori, pi� simili a macchine da scrivere, con sistemi di protezione fisici 
che 